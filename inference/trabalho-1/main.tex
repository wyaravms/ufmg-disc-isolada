\documentclass[a4paper, 11pt]{article}
\usepackage[utf8]{inputenc}
\usepackage[margin=1in]{geometry} 
\usepackage{amsmath,amsthm,amssymb}
\usepackage{listings}
\usepackage{graphicx}
\usepackage{indentfirst}
\renewcommand{\baselinestretch}{1}
\usepackage{subcaption}
\usepackage{float}
\usepackage{dsfont} %change indicator function
\graphicspath{ {images/} }
\usepackage{comment} % enables the use of multi-line comments (\ifx \fi) 
\usepackage{fullpage} % changes the margin

\begin{document}
%Header-Make sure you update this information!!!!
\noindent
{\Large\textbf{Trabalho 1} \hfill \\
Inferência \hfill Prof. Vinícius D. Mayrink \\
Wyara Vanesa Moura e Silva \hfill 25 de Outubro de 2022\\}

\section*{Questão}

Saiba que se $X_{i} \sim \mbox{Weibull}(\alpha, \beta)$ então

\begin{equation*}
f(x_{i}|\alpha, \beta) = \dfrac{\alpha}{\beta}\left(\dfrac{x_{i}}{\beta}\right)^{\alpha -1}\exp\left\{-\left(\dfrac{x_{i}}{\beta}\right)^{\alpha}\right\}, \hspace{0.5cm} \mbox{para} \hspace{0.5cm} x_{i}\geq 0.
\end{equation*}

Nesta distribuição contínua temos que $\alpha>0$ é o parâmetro de forma e $\beta>0$ é o parâmetro de escala.

Suponha que $X_{1}, \cdots, X_{n}$ seja uma amostra aleatória proveniente da distribuição Weibull$(\alpha,1)$. Implemente o método de Newton, usando o $software$ $R$, para obter a estimativa de máxima verossimilhança de $\alpha$. Utilize o conjunto de dados fornecido na página do curso.

\vspace{2ex}
\noindent
\textit{Solução:}\\

O gráfico da Figura \ref{fig1}, refere-se ao histograma dos dados fornecidos.

\begin{figure}[H]
\centering
\includegraphics[width = 0.7\textwidth, height =0.3\textheight]{Rplot06.pdf}
\caption{Histograma dos dados Weibull. A linha tracejada corresponde a densidade.}
\label{fig1}
\end{figure}

A função de verossimilhança, para $\alpha$ proveniente de uma amostra da distribuição Weibull$(\alpha,1)$, é:
\begin{equation*}
\begin{array}{lclll}
f(\textbf{x}|\alpha) & = & \displaystyle\prod_{i=1}^{n} \dfrac{\alpha}{1}\left(\dfrac{x_{i}}{1}\right)^{\alpha -1}\exp\left\{-\left(\dfrac{x_{i}}{1}\right)^{\alpha}\right\} \\
& = & \alpha^{n}\left[\displaystyle\prod_{i=1}^{n}\left(x_{i}\right)^{\alpha -1}\right]\exp\left\{-\displaystyle\sum_{i=1}^{n}\left(x_{i}\right)^{\alpha}\right\}  \\[10pt] 
\end{array}
\end{equation*}

E assim, a log-verossimilhança, função que queremos otimizar, é dada por:

\begin{equation*}
\begin{array}{lclll}
\ln f(\textbf{x}|\alpha) & = & n \ln\alpha + (\alpha -1)\displaystyle\sum_{i=1}^{n}\left(\ln x_{i}\right) -\displaystyle\sum_{i=1}^{n}\left(x_{i}\right)^{\alpha}  \\[10pt] 
\end{array}
\end{equation*}

A fim de utilizar o método de Newton para encontrar uma estimativa de máxima verossimilhança para $\alpha$, teremos que encontrar a primeira e segunda derivada da função log-verossimilhança. Desta forma, $f(\alpha)$ é a função que queremos encontrar as raizes utilizando o método de newton, e além dela também precisamos de $f'(\alpha)$ para implementar o algoritmo, tais funções são as respectivas derivadas da função log-verossimilhança:

\begin{equation*}
\begin{array}{lclclll}
f(\alpha) & = & \dfrac{\partial}{\partial\alpha}\ln f(\textbf{x}|\alpha) & = & \dfrac{n}{\alpha} + \displaystyle\sum_{i=1}^{n}\left(\ln x_{i}\right) -\displaystyle\sum_{i=1}^{n}(\ln x_{i})\left(x_{i}\right)^{\alpha}  \\[14pt] 
f'(\alpha) & = & \dfrac{\partial^2}{\partial\alpha^2}\ln
f(\textbf{x}|\alpha) & = & -\dfrac{n}{\alpha^2} -\displaystyle\sum_{i=1}^{n}(\ln x_{i})^{2}\left(x_{i}\right)^{\alpha}  \\[10pt] 
\end{array}
\end{equation*}

%Assim, $f(\alpha)$ é a função que queremos encontrar as raizes utilizando o método de newton, e além dela também precisamos de $f'(\alpha)$ para implementar o algoritmo, tais funções são as respectivas derivadas encontradas anteriormente:

%\begin{equation*}
%\begin{array}{lclll}
%f(\alpha) & = & \dfrac{n}{\alpha} + \displaystyle\sum_{i=1}^{n}\left(\ln x_{i}\right) -\displaystyle\sum_{i=1}^{n}\ln(x_{i})\left(x_{i}\right)^{\alpha}  \\[14pt] 
%f'(\alpha) & = & -\dfrac{n}{\alpha^2} -\displaystyle\sum_{i=1}^{n}\ln(x_{i})^{2}\left(x_{i}\right)^{\alpha}  \\[10pt] 
%\end{array}
%\end{equation*}

Para implementação do algoritmo no $R$ foi utilizado a fórmula de recorrência do método de newton:
\begin{equation*}
\begin{array}{lclll}
\alpha_{k+1} & = & \alpha_{k} - \dfrac{f(\alpha_{k})}{f'(\alpha_{k})}, \hspace{2ex} k=0,1,2,\cdots \\[10pt] 
\end{array}
\end{equation*}

Além disso, foi escolhido uma precisão de $10^{-7}$, ou seja, o critério de parada do programa será quando o valor absoluto da diferença entre o valor anterior e o atual forem menores que esta precisão. Outra característica importante do programa é a necessidade da entrada de um chute inicial para o parâmetro (ou seja, é necessário que esteja dentro do espaço paramétrico) que se deseja realizar a estimação, neste caso $\alpha$. 

Os códigos em $R$ estão na parte final deste trabalho, a partir daqui serão mostrados alguns resultados utilizando esses algoritmos. 

Para os dados fornecidos, o $\hat{\alpha}$ estimado que irá maximizar a função de verossimilhança será: $\hat{\alpha}$ $=$ $4.965997$. O chute inicial utilizado foi $\hat{\alpha}_{0}$ $=$ $10$.

A Tabela \ref{tab1} mostra alguns resultados de estimações do parãmetro $\alpha$  utilizando o algoritmo desenvolvido, foram geradas amostras com diferentes tamanhos de $n$, e com $\alpha = 5$. O chute inicial utilizado nestas simulações foram todos iguais, $\hat{\alpha}_{0}$ $=$ $2$.

\begin{table}[H]
\caption{Simulação para diferentes tamanhos de amostras.}\label{tab1}
\centering
\begin{tabular}{llllllcc}
\hline
n & 10 & 30 & 100 & 1000  \\
\hline
$\alpha$ & 5.595741 & 4.692904 & 5.088293 & 5.0562 \\
\hline
\end{tabular}
\end{table}

Podemos observar que quanto maior a amostra, o algoritmo tem um melhor desempenho quanto a estimação do verdadeiro valor para o parâmetro $\alpha$. Não foram obtidos problemas na estimação do parâmetro, quando foram introduzidos diferentes valores nos chutes iniciais de $\alpha$.

\newpage
\subsection*{Código R}

Veja abaixo os códigos em R utilizados.

\begin{verbatim}
rm(list=ls(all=TRUE))
set.seed(22)
setwd("C:\\Users\\Wyara Silva\\Meu Drive\\disciplina-isolada-ufmg-2022\\
inferencia-estatistica\\trabalho-1")

dados.f = read.table("Dados_Weibull.txt")
dados = unlist(dados.f, use.names = FALSE)

library(ggplot2)

ggplot(dados.f, aes(x=V1)) + 
  geom_histogram(aes(y=..density..), bins = 30, color="black", fill="grey")+
  geom_density(alpha=.2, linetype = 2, fill="light blue") +
  labs(x="dados", y = "densidade")

newton.raphson.w = function(x.dados, alpha.0=10, precisao = 1e-7, n=100){
  
  #primeira e segunda derivada da log-verossimilhança com relação a alpha
  dlogLikW = function(y){(length(x.dados)/y) + sum(log(x.dados)) -
  sum(log(x.dados)*(x.dados^y))}
  ddlogLikW = function(z){-(length(x.dados)/z^2) -
  sum((log(x.dados)^2)*(x.dados^z))}
  
  for (i in 1:n) {
    #equação de recorrência de newton-raphson
    alpha.1 = alpha.0 - dlogLikW(alpha.0)/ddlogLikW(alpha.0)
    
    #verificar se convergiu
    if(abs(alpha.1 - alpha.0) < precisao){
      res = list(alpha.1,i)
      names(res) = c("alpha.estimado","n.iter")
      return(res)
    }
    
    #nova iteração
    alpha.0 = alpha.1
  }
  print("com o número de iterações não houve convergência")
}

newton.raphson.w(dados)

#simulação
n = c(10,30,100,1000)

estimativas = array(NA, dim=c(1, length(n)))

for (i in 1:length(n)){
  #a parametrização da distribuição weibull na função 'rweibull'
  #é igual a utilizada neste trabalho.
  dat =  rweibull(n[i], shape=5, scale=1)
  estimativas[,i] = newton.raphson.w(dat,2)$alpha.estimado
}

estimativas

\end{verbatim}

\end{document}
